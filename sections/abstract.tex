This paper addresses some of the security challenges posed by AI agents, framing them as part of the broader problem of distributed systems. 
It draws parallels with enterprise patterns for distributed systems and aims to lay the foundation for a new authorization (AuthZ) protocol. 
The key insight is that tokens, while useful, are not sufficient: authorization must move beyond a token-centric model. 
Instead, what is needed is a new zero-trust-authz protocol that, when combined with tokens, can support the development of next-generation AuthZ systems.  

In practice, companies will continue to rely on standards such as OIDC and SAML, and OAuth will certainly remain an initiator for many flows. 
However, the real gap lies in what happens after token delivery — a space where existing protocols have paid little attention, but which is critical for the future of secure and dynamic authorization.  
The proposed model also leaves room for the integration of decentralized approaches, such as UCAN, ZCAP-LD, and other capability-based frameworks, enabling hybrid ecosystems that combine traditional standards with emerging decentralized authorization models.