
\lettrine{T}{he} Agentic Era is emerging in industry discussions, as many recognize that certain assumptions no longer hold and past trade-offs are becoming increasingly fragile.
Change can often be unsettling, especially when existing systems and business models are built on established approaches. 
Yet it may be time to start with a blank page and re-examine the compromises we have inherited.

\vspace{0.5em} The community should take this as a challenge to rethink the foundations of authorization, especially in the context of distributed systems and AI agents.

\vspace{0.5em} In the near future, for sure companies will continue to rely on OAuth to authenticate humans, and on solutions such as WIMSE or SPIFFE to authenticate non-human workloads. Yet at some point, we must stop patching legacy models and acknowledge that new architectures demand new foundations.

\vspace{0.5em} This new foundation can open entirely new markets. 
It can still leverage existing technologies, but with a fresh approach that enables innovation, new value creation, and the development of novel products. 
Industry has been built around identities, and most importantly this new foundation could enable the real adoption of AI agents within the enterprise.

\vspace{0.5em} The authentication layer is relatively solid, whether using centralized or decentralized approaches, but authorization still works only under certain trade-offs. These trade-offs cannot support AI agents and, more generally, distributed systems. 

\vspace{0.5em} History of science teaches us that progress often comes from looking across disciplines, borrowing solutions, and applying them in new contexts, that is what this paper aims to do looking at Enterprise Patterns for distributed systems.
It is also important to consider this in the context of Zero Trust Network Access (ZTNA) and, more broadly, the Zero Trust model, which will be crucial for the future of secure access.

\vspace{0.5em} This paper adopts such an approach, challenging the core trade-off that the industry has relied on: the identity and HTTP-centric model.
An AI agent protocol would need to include many other features; however, this paper focuses specifically on authorization (authz).