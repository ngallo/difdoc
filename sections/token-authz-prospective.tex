Tokens have worked very well for \textbf{authentication (AuthN)}.  
An \textbf{authentication token} simply states: \emph{``I am Nicola''}.  
If this statement is true now, it will remain true ever from now.  

However, there are essentially two problems:
\begin{enumerate}
    \item \textbf{Trust}: anyone can claim such a statement.  
    To address this, tokens must be \textbf{signed}, giving rise to mechanisms such as \textbf{JWTs}.
    \item \textbf{Replay}: anyone can replay the same token.  
    The only option to mitigate this risk is to add an \textbf{expiration time}.
\end{enumerate}

These are \textbf{trade-offs} and represent inherent \textbf{limits of security}.  
In practice, tokens exploit a \textbf{PACELC-style trade-off}: in the absence of partitions, they take advantage of the \textbf{latency–consistency} trade-off.  

\begin{boxF}
    The \textbf{PACELC theorem} \cite{c9} states that in case of \textbf{network partitioning (P)} in a \textbf{distributed system}, one has to choose between \textbf{availability (A)} and \textbf{consistency (C)} (as per the CAP theorem), but else (E), even when the system is running normally in the absence of partitions, one has to choose between \textbf{latency (L)} and \textbf{loss of consistency (C)}.
\end{boxF}

This trade-off becomes dangerous when tokens represent \textbf{authorization (AuthZ)}.  
For example, \emph{``Nicola is entitled to a certain role now''} may be true at one moment 
but false only a few seconds later.  
Here \textbf{tokens become fragile}: while they have worked well for \textbf{authentication}, 
it is a mistake to assume they must also be the \textbf{golden standard for authorization}.  

Below we present a more formal representation of this problem.

\subsection{Temporal Validity and Risk}
Let $T_a$ be the validity period of an \textbf{authentication token} and $T_z$ the validity of an \textbf{authorization}.  
To ensure security:
\[
T_z \ll T_a, \quad \text{ideally } T_z \to 0
\]
In practice, $T_z$ should be kept as short as feasible, though operational trade-offs may apply.  

The \textbf{risk of compromise} grows with validity:
\[
R(T) = f(T), \quad f'(T) > 0
\]
Thus, \textbf{long-lived static authorization tokens} increase exposure.

\subsection{Transitive Delegation}
A \textbf{delegation} between $A$ and $B$ is denoted:
\[
D_i : A \rightarrow B
\]
A chain of $n$ delegations is:
\[
D_1 \circ D_2 \circ \dots \circ D_n
\]

If a token explicitly encodes a single linear chain of $n$ delegations, 
its size grows linearly:
\[
O(n)
\]

However, when delegation supports \textbf{branching}, \textbf{conditions}, or \textbf{attenuation}, 
the number of possible valid paths increases combinatorially. 
Encoding all possible transitive delegations into a \textbf{static token} leads to exponential growth in the worst case:
\[
O(2^n)
\]

This is not scalable. Therefore, \textbf{authorization must be reconstructed dynamically at request time}.

\subsection{Context and Runtime Validity}
\textbf{System state} evolves with time:
\[
S(t) = \{E, W, P\}
\]
where $E$ are \textbf{entities}, $W$ \textbf{workloads}, and $P$ \textbf{policies}.  

At request time $t_r$, validity requires:
\[
\exists p \in P \;\; \text{valid for } (e,w) \mid t=t_r
\]
\textbf{Static tokens} cannot guarantee consistency with the \textbf{live system state} $S(t_r)$.

\subsection{Trust Model}
\textbf{Authorization} is a function of the \textbf{live context}:
\[
T(E, W, P, t) =
\begin{cases}
1 & \text{if authorized at time $t$}\\
0 & \text{otherwise}
\end{cases}
\]

\textbf{Static tokens} implicitly encode \textbf{authorization state} at issuance time $t_0$, 
and reuse assumes that the same decision holds for all later times:
\[
T(E,W,P,t_0) = T(E,W,P,t), \; \forall t>t_0
\]
which is false in \textbf{dynamic systems}.

\subsection{Conclusion}
Token-based authorization is:
\begin{enumerate}
\item \textbf{Insecure}: $R(T)$ increases with $T$.
\item \textbf{Non-scalable}: \textbf{delegation graphs} may induce combinatorial explosion (up to $O(2^n)$).
\item \textbf{Inconsistent}: $S(t) \neq S(t_0)$ in \textbf{dynamic systems}.
\end{enumerate}
Therefore, \textbf{authorization must be computed on-demand}, consistent with \textbf{Zero Trust principles}.

\subsection{Practical Considerations}
The model above is a \textbf{theoretical ideal}; in practice it is almost impossible to achieve perfectly. 
As in many areas of \textbf{security}, trade-offs must be made to balance \textbf{feasibility} and \textbf{protection}.  
What matters is not the invention of a \textbf{``better token''} but the design of a more \textbf{sophisticated authorization engine} capable of assessing \textbf{multiple sources of trust} --- including \textbf{tokens}, 
\textbf{capabilities} (e.g.\ zcaps), \textbf{contextual signals}, and other \textbf{dynamic inputs} --- in a \textbf{best-effort approach}.  

The key point is that the \textbf{next generation of authorization protocols} cannot remain \textbf{token-centric}. 
Instead, \textbf{authorization} must emerge from a richer machinery that integrates \textbf{diverse identities} 
and \textbf{trust sources}, and can even operate \emph{without} \textbf{tokens} altogether.
